
\chapter{Evaluation}\label{sec:evaluation}

This chapter is aimed to demonstrate that applying LETOR technique improves the effectiveness of the expert search system. 
It separates into 2 experiments. The first experiment does not make use of K-fold Cross-validation technique (see Section~\ref{sec:withoutLETOR})
but the second experiment does (see Section~\ref{sec:withLETOR}).
We use LETOR technique proposed in Chapter~\ref{sec:letor}. Basically, there are 2 necessary datasets: training and testing datasets and 1 optional dataset:
validation dataset. In IR, we refer them as training, testing and validation queries respectively. 
Some of the queries were provided by Dr. Craig Macdonald and the others were made up by myself. 
We made an assumption that the current system (using only publication as expertise evidence) is effective. This means that relevance judgements 
(the experts retrieved in response to a query) can be obtained manually from the current system. To make relevance judgements, top 20 experts retrieved in response to a query
are examined if they are truly an expert by visiting their official home page and with the help of Dr. Craig Macdonald.
Since constructing relevance judgements is time-consuming, we have made only 67 queries. 
Throughout this chapter, the queries annotated by * denote queries whose relevant experts are restricted to those experts only from 
the University of Glasgow. It is difficult to judge experts not in the University of Glasgow because we 
do not know them. However, in practice, this will incur dataset quality~\cite{craig}.

Before we discuss evaluation, we should have some goals set up. This is necessary because the evaluation results can be compared against the goals in order to
conclude whether all goals are achieved or not. As stated in Chapter~\ref{section:aims}, our goal is to integrate new kind of expertise evidence, funded projects,
with existing expertise evidence, publications, to enhance the effectiveness of the expert search system. At the end of this chapter, we have to be able to 
answer the following question:

\begin{quotation}
 \textit{Does integrating funded projects with publications using learning to rank technique (LETOR) help improve the effectiveness 
 of the expert search system?}
\end{quotation}

We refer to a currently deployed SICSA expert search system (without Learning to Rank) as our \textit{baseline}. This term will be frequently used 
throughout this chapter.

\section{Evaluation Without K-fold Cross-validation}\label{sec:withoutLETOR}
This section is aimed to demonstrate that applying LETOR technique improves the effectiveness of the deployed expert search system. 
K-fold Cross-validation technique is not applied in this section.

\subsection{Experimental Setting}
 The following is a list of steps in running an evaluation:
\begin{enumerate}
 \item all the queries are split into testing and training queries.
 \item from testing queries, generate a qrels file as discussed in Chapter~\ref{sec:treceval}.
 \item generate 2 learned models (see Chapter~\ref{section:producelearnedmodel}) from training LETOR file using AdaRank, and Coordinate Ascent algorithms.
 \item apply each learned model to all testing queries.
 \item for each learned model, apply it to LETOR file of testing queries using RankLib and generate a result file in TREC format (see Chapter~\ref{sec:treceval}) 
 \item evaluate the result file in TREC format with the qrels file for each test using trec\_eval (see Chapter~\ref{sec:treceval}).
\end{enumerate}

To evaluate our baseline, we can set other feature values apart from publication query feature value to 0 and perform step (5).

\subsubsection{Training Queries}
34 queries shown in table~\ref{table:trainingqueries} are used for training. 

\begin{table}
\centering
\scalebox{0.5}{\begin{tabular}{|c|l|c|l|}
\hline \textbf{\#} & \textbf{Query} & \textbf{\#} & \textbf{Query} \\
\hline 1 & *language modelling & 18 & game theory\\
\hline 2 & *manets & 19 & stable marriage \\
\hline 3  & *match & 20 & quantum computation\\ 
\hline 4  & *multimodal & 21 & constraint modelling\\ 
\hline 5  & *music as navigation cues & 22 & home networks\\ 
\hline 6  & networking security & 23 & wireless sensor networks\\ 
\hline 7  & *neural network & 24 & distributed systems\\ 
\hline 8  & *older adults use of computers & 25 & operating system\\ 
\hline 9  & *parallel logic programming & 26 & *terrier\\ 
\hline 10  & *query expansion & 27 & *text searching\\ 
\hline 11  & *road traffic accident statistics & 28 & *trec collection class\\ 
\hline 12  & *shoogle & 29 & *usability\\ 
\hline 13  & *skill-based behavior & 30 & *utf support terrier\\ 
\hline 14  & *sound in multimedia human-computer interfaces & 31 & *visual impairment\\ 
\hline 15  & statistical inference & 32 & *wafer fab cost \\ 
\hline 16  & *suffix tree & 33 & database\\ 
\hline 17  & mobile hci & 34 & programming languages\\ 
\hline
\end{tabular}
}
\caption{Training Queries} \label{table:trainingqueries}
\end{table}

\subsubsection{Testing Queries}
33 queries are used for testing as shown in table~\ref{table:testingqueries}.
\begin{table}
\centering
\scalebox{0.5}{\begin{tabular}{|c|l|c|l|}
\hline \textbf{\#} & \textbf{Query} & \textbf{\#} & \textbf{Query} \\
\hline 1 & *empirical methods & 18 & *3d human body segmentation\\
\hline 2 & *eurosys 2008 & 19 & *anchor text \\
\hline 3  & *facial reconstruction & 20 & *clustering algorithms\\ 
\hline 4  & *force feedback & 21 & *different matching coefficients\\ 
\hline 5  & functional programming & 22 & *discrete event simulation\\ 
\hline 6  & *glasgow haskell compiler & 23 & *discrete mathematics\\ 
\hline 7  & *grid computing & 24 & *distributed predator prey\\ 
\hline 8  & artificial intelligence & 25 & *divergence from randomness\\ 
\hline 9  & computer graphics & 26 & *ecir 2008 \\
\hline 10  & robotics & 27 & *group response systems\\ 
\hline 11  & data mining & 28 & *handwriting pin\\ 
\hline 12  & computer vision & 29 & *haptics\\ 
\hline 13  & information security & 30 & *haptic visualisation\\ 
\hline 14  & *3d audio & 31 & *hci issues in mobile devices\\ 
\hline 15  & expert search & 32 & *human error health care \\ 
\hline 16  & information retrieval & 33 & *information extraction\\ 
\hline 17  & machine translation &  & \\ 
\hline
\end{tabular}
}
\caption{Testing Queries} \label{table:testingqueries}
\end{table}

\subsection{Experimental Results}

\begin{table}
\centering
\scalebox{0.5}{\begin{tabular}{|c|c|c|c|c|}
\hline \textbf{Algorithm} & \textbf{MAP} & \textbf{NDCG} & \textbf{MRR} & \textbf{Number of Relevant Experts}\\
\hline AdaRank & 0.0153 & 0.1872 & 0.0116 & 106\\
\hline Coordinate Ascent & 0.5264 & 0.6618 & 0.6332 & 106 \\
\hline Baseline  & 0.5499 & 0.6871 & 0.6484 & 106\\ 
\hline
\end{tabular}
}
\caption{Evaluation Results} \label{table:evaluationresult}
\end{table}

Table~\ref{table:evaluationresult} shows the retrieval effectiveness of the proposed LETOR techniques and baseline. It can be clearly seen
that overall the baseline gives better performance compared to using LETOR considering MAP, NDCG and MRR evaluation metrics. Coordinate Ascent gives second best 
performance and AdaRank gives the worst performance. Therefore, the answer to the question is NO, using LETOR technique (without K-fold Cross-validation) does not enhance the 
effectiveness of the deployed expert search system.

\begin{table}
\centering
\scalebox{0.5}{\begin{tabular}{|l|c|c|c|c|c|c|}
\hline \textbf{Testing Queries} & \multicolumn{2}{|c|}{\textbf{AdaRank}} & \multicolumn{2}{|c|}{\textbf{Coordinate Ascent}} & \multicolumn{2}{|c|}{\textbf{Baseline}} \\
\hline 				 & \textbf{MAP} & \textbf{MRR}  & \textbf{MAP} & \textbf{MRR}  & \textbf{MAP} & \textbf{MRR} \\
\hline *3d audio & 0.0053 & 0.0053 & 1.0000 & 1.0000 & 1.0000 & 1.0000 \\
\hline *3d human body segmentation & 0.0127 & 0.0068 & 0.1452 & 0.2000 & 0.1824 & 0.2000 \\
\hline *anchor text & 0.0263 & 0.0088 & 0.7894 & 1.0000 & 0.7894 & 1.0000 \\
\hline artificial intelligence & 0.0344 & 0.0233 & 0.5944 & 1.0000 & 0.5972 & 1.0000 \\
\hline *clustering algorithms & 0.0166 & 0.0071 & 0.3048 & 0.2000 & 0.3048 & 0.2000 \\
\hline computer graphics & 0.0047 & 0.0047 & 0.5000 & 0.5000 & 0.5000 & 0.5000 \\
\hline computer vision & 0.0142 & 0.0130 & 0.4611 & 0.5000 & 0.4611 & 0.5000 \\
\hline data mining & 0.0181 & 0.0118 & 0.9167 & 1.0000 & 0.9167 & 1.0000 \\
\hline *different matching coefficients & 0.0091 & 0.0063 & 0.5833 & 0.5000 & 0.5833 & 0.5000 \\
\hline *discrete event simulation & 0.0078 & 0.0078 & 0.3333 & 0.3333 & 0.3333 & 0.3333 \\
\hline *discrete mathematics & 0.0164 & 0.0115 & 0.0306 & 0.0400 & 0.1394 & 0.1250 \\
\hline *distributed predator prey & 0.0058 & 0.0058 & 0.0037 & 0.0037 & 0.1111 & 0.1111 \\
\hline *divergence from randomness & 0.0124 & 0.0058 & 0.9167 & 1.0000 & 0.9167 & 1.0000 \\
\hline *ecir 2008 & 0.0152 & 0.0070 & 0.8125 & 1.0000 & 0.8125 & 1.0000 \\
\hline *empirical methods & 0.0189 & 0.0072 & 0.2092 & 0.1667 & 0.2154 & 0.1667 \\
\hline *eurosys 2008 & 0.0055 & 0.0055 & 1.0000 & 1.0000 & 1.0000 & 1.0000 \\
\hline expert search & 0.0107 & 0.0075 & 0.5037 & 1.0000 & 0.5227 & 1.0000 \\
\hline *facial reconstruction & 0.0039 & 0.0039 & 0.0021 & 0.0021 & 0.2000 & 0.2000 \\
\hline *force feedback & 0.0073 & 0.0050 & \textbf{0.6429} & 1.0000 & 0.6250 & 1.0000 \\
\hline functional programming & 0.0244 & 0.0112 & 0.1687 & 0.2000 & 0.2038 & 0.1429 \\
\hline *glasgow haskell compiler & 0.0101 & 0.0101 & 0.3333 & 0.3333 & 0.5000 & 0.5000 \\
\hline *grid computing & 0.0123 & 0.0083 & 0.3095 & 0.3333 & 0.3333 & 0.3333 \\
\hline *group response systems & 0.0054 & 0.0054 & 0.5000 & 0.5000 & 0.5000 & 0.5000 \\
\hline *handwriting pin & 0.0054 & 0.0038 & 1.0000 & 1.0000 & 1.0000 & 1.0000 \\
\hline *haptic visualisation & 0.0224 & 0.0102 & \textbf{0.6111} & 1.0000 & 0.5861 & 1.0000 \\
\hline *haptics & 0.0407 & 0.0133 & 0.4929 & 0.5000 & 0.5819 & 0.5000 \\
\hline *hci issues in mobile devices & 0.0268 & 0.0091 & 0.5821 & 1.0000 & 0.6160 & 1.0000 \\
\hline *human error health care & 0.0064 & 0.0064 & \textbf{0.5821} & \textbf{1.0000} & 0.3333 & 0.3333 \\
\hline *information extraction & 0.0069 & 0.0050 & \textbf{0.3333} & \textbf{0.3333} & 0.2500 & 0.2500 \\
\hline information retrieval & 0.0308 & 0.0102 & \textbf{0.7496} & 1.0000 & 0.7468 & 1.0000 \\
\hline information security & 0.0163 & 0.0120 & 0.7333 & 1.0000 & 0.7333 & 1.0000 \\
\hline machine translation & 0.0117 & 0.0118 & 1.0000 & 1.0000 & 1.0000 & 1.0000 \\
\hline robotics & 0.0404 & 0.1111 & \textbf{0.5565} & 1.0000 & 0.5523 & 1.0000\\
\hline

\end{tabular}
}
\caption{Evaluation Results of Each Testing Query} \label{table:eachqueryevaluationresult}
\end{table}

Table~\ref{table:eachqueryevaluationresult} shows the evaluation results of each testing query. Numbers in \textbf{bold}
are numbers higher than ones in the baseline. It has shown that applying LETOR technique using
Coordinate Ascent algorithm to the testing queries shown in table~\ref{table:baselineQ} outperformed baseline ones considering MAP/MRR values.
Experts in \textbf{bold} indicates that using Coordinate Ascent they are ranked higher higher than baseline. N/A indicates that the experts 
are not in the top 10 of the ranking. Experts with 0 graded relevance value are not shown in this table. Graded relevance values of experts can be 
found in Appendix Chapter.

\begin{table}
\centering
 \scalebox{0.5}{
\begin{tabular}{|l|l|c|c|c|}
  \hline \textbf{Testing Queries} & \multicolumn{1}{|c|}{\textbf{Expert}} & \textbf{Grade} & \textbf{Rank (Coordinate Ascent)} & \textbf{Rank (Baseline)} \\
 
    \hline *force feedback & Stephen Brewster & 2 & 1 & 1 \\
   \hline  		 & Roderick Murray-Smith & 2 & N/A & N/A \\
   \hline *human error health care & Stephen Robert & 1 & 3 & 3 \\
   \hline *information extraction & Joemon Jose & 2 & 8 & 8 \\
   \hline 			& Alessandro Vinciarelli & 1 & 4 & 4 \\
   \hline *haptic visualisation & Stephen Brewster & 2 & 1 & 1 \\
   \hline 			& Roderick Murray-Smith & 2 & N/A & 5 \\
   \hline 			& \textbf{John Williamson} & 2 & 4 & 9 \\
   \hline 			& Phillip Gray & 2 & N/A & 6 \\
   \hline information retrieval & Joemon Jose (Glasgow University) & 2 & 1 & 1 \\
   \hline 			& Cornelis Van Rijsbergen (Glasgow University) & 2 & 2 & 2 \\
   \hline 			& Craig Macdonald (Glasgow University) & 1 & 4 & 4 \\
   \hline 			& Iadh Ounis (Glasgow University) & 2 & 5 & 5 \\
   \hline 			& Leif Azzopardi (Glasgow University) & 1 & 6 & 6 \\
   \hline 			& Victor Lavrenko (Edinbugh University) & 1 & 8 & 8 \\
   \hline robotics & Sethu Vijayakumar (Edinbugh University) & 2 & 1 & 1 \\
   \hline	    & Subramanian Ramamoorthy (Edinbugh University) & 1 & 7 & 7 \\
   \hline 	    & Andrew Brooks (Dundee University) & 1 & 4 & 4 \\
   \hline 	    & Clare Dixon (Glasgow University) & 1 & 9 & 9 \\
   \hline 	    & Katia Sycara (Aberdeen University) & 2 & 10 & 10 \\
   \hline
               \end{tabular}
}
\caption{Ranking Evaluation}\label{table:baselineQ}
\end{table}

According to table~\ref{table:baselineQ}, only 1 expert \textit{John Williamson} retrieved with respect to testing query, \textit{haptic visualisation}, 
is ranked better than baseline.

\begin{itemize}
 \item force feedback
 \item haptic visualisation
 \item human error heath care
 \item information extraction
 \item information retrieval
 \item robotics
\end{itemize}

\section{Evaluation With K-fold Cross-validation}\label{sec:withLETOR}
In the previous section, the evaluation results have shown that applying LETOR makes the deployed expert search system perform poorer. 
This section aims to evaluate the effectivenes of the system by applying K-fold Cross-validation
technique described in Chapter~\ref{sec:kfold}. We decided to choose $K = 5$. This is fairly normal for learning to rank~\cite{craig}.
Higher K is possible, but that infers that the LETOR needs more training. RankLib provides K-fold Cross-validation functionality. However, we will not
use this functionality because some of our datasets results were restricted to those experts only from the University of Glasgow. For this reason, the datasets should
be split into training, validation and testing queries manually to avoid one of them having only experts from University of Glasgow~\cite{craig}.

\subsection{Experimental Setting}

As discussed earlier, we decided to choose $K = 5$ for K-fold Cross-validation. Each of the 5 folds has all of the queries: 60\% for training, 
20\% for validation and 20\% for testing~\cite{kfoldcv}. Across the 5 folds, each query will appear 3 times in training, once in validation and 
once in testing queries. For each fold, training queries are trained and validated by validation queries to obtain a learned model.
Then a learned model for each fold is applied to testing queries of that fold. Results of each run across the 5 folds are combined into one results file in
TREC format. With the qrels file, this results file is then evaluated with trec\_eval. Finally, one of the learned models is used by the system.
Testing, training and validation queries for each fold can be found on Appendix.

% \subsubsection{Fold 1}
% Tables~\ref{table:fold1testing},~\ref{table:fold1training} and~\ref{table:fold1validating} show testing, training and validation queries respectively.
% \begin{table}
% \centering
% 
% \caption{Testing Queries for Fold 1} \label{table:fold1testing}
% \end{table}

% \begin{table}
% \centering
% 
% \caption{Validation Queries for Fold 1} \label{table:fold1validating}
% \end{table}

% \subsubsection{Fold 2}
% Tables~\ref{table:fold2testing},~\ref{table:fold2training} and~\ref{table:fold2validating} show testing, training and validation queries respectively.

% \begin{table}
% \centering
% 
% \caption{Testing Queries for Fold 2} \label{table:fold2testing}
% \end{table}

% \begin{table}
% \centering
% 
% \caption{Validation Queries for Fold 2} \label{table:fold2validating}
% \end{table}

% \subsubsection{Fold 3}
% Tables~\ref{table:fold3testing},~\ref{table:fold3training} and~\ref{table:fold3validating} show testing, training and validation queries respectively.


% \begin{table}
% \centering
% 
% \caption{Testing Queries for Fold 3} \label{table:fold3testing}
% \end{table}

% \begin{table}
% \centering
% 
% \caption{Validation Queries for Fold 3} \label{table:fold3validating}
% \end{table}

% \subsubsection{Fold 4}
% Tables~\ref{table:fold4testing},~\ref{table:fold4training} and~\ref{table:fold4validating} show testing, training and validation queries respectively.


% \begin{table}
% \centering
% 
% \caption{Testing Queries for Fold 4} \label{table:fold4testing}
% \end{table}

% \begin{table}
% \centering
% 
% \caption{Validation Queries for Fold 4} \label{table:fold4validating}
% \end{table}

% \subsubsection{Fold 5}
% 
% Tables~\ref{table:fold5testing},~\ref{table:fold5training} and~\ref{table:fold5validating} show testing, training and validation queries respectively.

\subsection{Experimental Results}

Table~\ref{table:kfoldevaluationresult} shows the retrieval effectiveness of the proposed LETOR techniques using K-fold cross-validation techinque
and retrieval effectiveness of the baseline.
It can be clearly seen that overall the baseline is still better than LETOR using K-fold cross-validation. 
Coordinate Ascent still gives second best performance. But using K-fold cross-validation technique, the effectiveness is slightly better than
not using K-fold cross-validation (discussed in the last section). Therefore, the answer
to the question is still NO, using LETOR technique with K-fold cross-validation does not enhance the effectiveness of the deployed expert search system.
Table~\ref{table:kfoldqueryresult} shows the evaluation results of each testing query. Numbers in \textbf{bold} show testing queries applying Coordinate Ascent 
whose evaluation metrics are higher
than baseline ones. Numbers underlined show that testing queries applying AdaRank algorithm perform better than ones applying Coordinate Ascent.
The evaluation results of the testing queries applied with LETOR using Coordinate Ascent shown in table~\ref{table:kfoldQ} outperform 
baseline ones considering MAP/MRR values. Experts shown in table~\ref{table:kfoldQ} in the top 10 of the ranking are analysed to if the ranking
with LTR using K-fold Cross-validation is better than baseline. Experts with 0 graded relevance value are not shown in this table. 
Graded relevance values of experts can be found in Appendix Chapter. According to table~\ref{table:baselineQ}, applying K-fold Cross-validation change the ranking compared to the baseline.
\begin{table}
\centering
 \scalebox{0.5}{
\begin{tabular}{|l|l|c|c|c|}
  \hline \textbf{Testing Queries} & \multicolumn{1}{|c|}{\textbf{Expert}} & \textbf{Grade} & \textbf{Rank (Coordinate Ascent)} & \textbf{Rank (Baseline)} \\
  \hline artificial intelligence & David Bell (Strathclyde University) & 1 & 1 & 1 \\
  \hline 			& Ruth Aylett (Heriot Watt University) & 2 & 2 & 2 \\
  \hline 			& Alan Smaill (Edinburgh University) & 1 & 7 & 7 \\
  \hline 			& Alan Bundy (Edinburgh University) & 1 & 9 & 9 \\
  \hline *ecir 2008 & Iadh Ounis & 2 & 2 & 2 \\
  \hline 	    & Craig Macdonald & 2 & 1 & 1 \\
  \hline 	    & Joemon Jose & 2 & 8 & 8 \\
  \hline *force feedback & Stephen Brewster & 2 & 1 & 1 \\
   \hline  		 & Roderick Murray-Smith & 2 & 8 & 8 \\ 
   \hline information retrieval & Joemon Jose (Glasgow University) & 2 & 1 & 1 \\
   \hline 			& Cornelis Van Rijsbergen (Glasgow University) & 2 & 2 & 2 \\
   \hline 			& Craig Macdonald (Glasgow University) & 1 & 4 & 4 \\
   \hline 			& Iadh Ounis (Glasgow University) & 2 & 5 & 5 \\
   \hline 			& Leif Azzopardi (Glasgow University) & 1 & 6 & 6 \\
   \hline 			& Victor Lavrenko (Edinbugh University) & 1 & 8 & 8 \\
   \hline
               \end{tabular}
}
\caption{Ranking Evaluation using K-fold Cross-validation}\label{table:kfoldQ}
\end{table}



\begin{table}
\centering
\scalebox{0.5}{\begin{tabular}{|c|c|c|c|c|}
\hline \textbf{Algorithm} & \textbf{MAP} & \textbf{NDCG} & \textbf{MRR} & \textbf{Number of Relevant Experts}\\
\hline AdaRank & 0.0469 & 0.2397 & 0.0843 & 106 \\
\hline Coordinate Ascent & 0.5321 & 0.6703 & 0.6366 & 106 \\
\hline Without LETOR  & 0.5499 & 0.6871 & 0.6484 & 106 \\ 
\hline
\end{tabular}
}
\caption{Evaluation Results Using K-fold Cross-validation} \label{table:kfoldevaluationresult}
\end{table}

\begin{table}
\centering
\scalebox{0.5}{\begin{tabular}{|l|c|c|c|c|c|c|}
\hline \textbf{Testing Queries} & \multicolumn{2}{|c|}{\textbf{AdaRank}} & \multicolumn{2}{|c|}{\textbf{Coordinate Ascent}} & \multicolumn{2}{|c|}{\textbf{Without LETOR}} \\
\hline 				 & \textbf{MAP} & \textbf{MRR}  & \textbf{MAP} & \textbf{MRR}  & \textbf{MAP} & \textbf{MRR} \\
\hline 3d audio & 0.0227 & 0.0227 & 1.0000 & 1.0000 & 1.0000 & 1.0000\\
\hline 3d human body segmentation & 0.0239 & 0.0222 & 0.1824 & 0.2000 & 0.1824 & 0.2000 \\
\hline anchor text & 0.0263 & 0.0088 & 0.7894 & 1.0000 & 0.7894 & 1.0000 \\
\hline artificial intelligence & 0.0344 & 0.0233 & \textbf{0.6064} & 1.0000 & 0.5972 & 1.0000 \\
\hline clustering algorithms & 0.0166 & 0.0071 & 0.3048 & 0.2000 & 0.3048 & 0.2000 \\
\hline computer graphics & 0.1429 & 0.1429 & 0.5000 & 0.5000 & 0.5000 & 0.5000 \\
\hline computer vision & 0.0430 & 0.0667 & 0.4611 & 0.5000 & 0.4611 & 0.5000 \\
\hline data mining & 0.0181 & 0.0118 & 0.9167 & 1.0000 & 0.9167 & 1.0000 \\
\hline different matching coefficients & 0.0076 & 0.0052 & 0.5833 & 0.5000 & 0.5833 & 0.5000 \\
\hline discrete event simulation & 0.0078 & 0.0078 & 0.3333 & 0.3333 & 0.3333 & 0.3333 \\
\hline discrete mathematics & 0.0164 & 0.0115 & 0.1214 & 0.1000 & 0.1394 & 0.1250 \\
\hline distributed predator prey & \uline{0.0058} & \uline{0.0058} & 0.0038 & 0.0038 & 0.1111 & 0.1111 \\
\hline divergence from randomness & 0.0205 & 0.0122 & 0.9167 & 1.0000 & 0.9167 & 1.0000 \\
\hline ecir 2008 & 0.0152 & 0.0070 & \textbf{0.8304} & 1.0000 & 0.8125 & 1.0000 \\
\hline empirical methods & 0.0316 & 0.0286 & 0.2154 & 0.1667 & 0.2154 & 0.1667 \\
\hline eurosys 2008 & 0.0055 & 0.0055 & 1.0000 & 1.0000 & 1.0000 & 1.0000 \\
\hline expert search & 0.0106 & 0.0076 & 0.5040 & 1.0000 & 0.5227 & 1.0000 \\
\hline facial reconstruction & 0.0270 & 0.0270 & 0.1429 & 0.1429 & 0.2000 & 0.2000 \\
\hline force feedback & 0.5141 & 1.0000 & \textbf{0.6429} & 1.0000 & 0.6250 & 1.0000 \\
\hline functional programming & 0.0244 & 0.0112 & 0.0952 & 0.1111 & 0.2038 & 0.1429 \\
\hline glasgow haskell compiler & 0.0101 & 0.0101 & 0.3333 & 0.3333 & 0.5000 & 0.5000 \\
\hline grid computing & 0.0123 & 0.0083 & 0.3333 & 0.3333 & 0.3333 & 0.3333 \\
\hline group response systems & 0.0047 & 0.0047 & 0.5000 & 0.5000 & 0.5000 & 0.5000 \\
\hline handwriting pin & 0.0422 & 0.0400 & 1.0000 & 1.0000 & 1.0000 & 1.0000 \\
\hline haptic visualisation & 0.0328 & 0.0400 & 0.5861 & 1.0000 & 0.5861 & 1.0000 \\
\hline haptics & 0.0331 & 0.0104 & 0.4091 & 0.5000 & 0.5819 & 0.5000 \\
\hline hci issues in mobile devices & 0.0319 & 0.0357 & 0.6160 & 1.0000 & 0.6160 & 1.0000 \\
\hline human error health care & 0.0064 & 0.0064 & 0.3333 & 0.3333 & 0.3333 & 0.3333 \\
\hline information extraction & 0.0254 & 0.0217 & 0.2500 & 0.2500 & 0.2500 & 0.2500 \\
\hline information retrieval & 0.2538 & 1.0000 & \textbf{0.7622} & \textbf{1.0000} & 0.7468 & 0.7468 \\
\hline information security & 0.0275 & 0.0476 & 0.7333 & 1.0000 & 0.7333 & 1.0000 \\
\hline machine translation & 0.0117 & 0.0118 & 1.0000 & 1.0000 & 1.0000 & 1.0000 \\
\hline robotics & 0.0404 & 0.1111 & 0.5523 & 1.0000 & 0.5523 & 1.0000 \\
\hline
\end{tabular}
}
\caption{Evaluation Results of Each Testing Query Using K-fold Cross-validation} \label{table:kfoldqueryresult}
\end{table}



\section{Causes of Learning to Rank Failure}
It has shown in the last section that applying LETOR technique fails. The followings are reasons behind it~\cite{craig}:
\begin{itemize}
 \item The size of training/validation dataset is not big enough.
 \item Selected LETOR Features are not very useful.
 \item The number of publications and funded projects is not balanced. We have 22225 publications but only 369 funded projects. This results in 
 features related to publications outweighing funded project features in both query dependent and independent features.
\end{itemize}
For AdaRank, the performance is considerably poorer than Coordinate Ascent because AdaRank requires a large set of validation data to avoid overfitting~\cite{craig}.
Therefore, we decided to apply K-fold Cross-validation technique as it gave better performance compared to not applying K-fold Cross-validation technique
(see last section).
