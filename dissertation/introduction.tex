

\section{Introduction}

\subsection{Definition of Mining Academic Expertise from Funded Research (Expert Search System)}
In an expert search task, the users' need is to identify people who have relevant expertise to a topic of interest.
An expert search system is an Information Retrieval(IR) system that can aid users with their ``expertise need''. 
Moreover, it predicts and ranks the expertise of a set of candidate persons with respect to the users' query.
Mining academic Expertise from funded research is to develop tools for funded research data which can extract, analyse and retrieve those data.

\subsection{Why Implement one?}
With the advent of the vast pools of information and documents in large enterprise organisations, collaborative users regularly have the need to find 
not only documents, but also people with whom they share common interests, or who have specific knowledge in a required area. The system makes use of
textual evidence of expertise to rank candidates.

\subsection{Aims}
http://experts.sicsa.ac.uk/ is an existing academic search engine that assists in identifying the relevant experts within Scottish Universities, 
based on their recent publication output. The aim of this project is to develop mining tools for the data (funded research), 
and research ways to integrate it with existing deployed academic search engines to obtain the most effective search results. 
Learning to Rank Algorithm for Information Retrieval (IR) is also used in this project in order to enhance the effectiveness of the system.
Funded projects from Grant on the Web are obtained from 
- http://gow.epsrc.ac.uk/ and Research Councils UK - http://gtr.rcuk.ac.uk/. Academic funded projects and publications are used as expertise evidence 
to assist in identifying the relevant experts with respect to a user query.

\subsection{Prerequisites}
Where possible all technical concepts used within this paper will be clearly defined. However, to fully understand the remainder of this report, 
the reader should have at least some knowledge of the following concepts:
\begin{itemize}
 \item Object-Oriented programming concepts (preferably in Java)
 \item Concepts of Information Retrieval (IR)
 \item Basics of Learning to Rank Algorithms for IR (preferably AdaRank)
\end{itemize}




