


\chapter{Introduction}

\section{The Scottish Informatics and Computer Science Alliance (SICSA)}

``The Scottish Informatics and Computer Science Alliance (SICSA) is a collaboration of Scottish Universities whose goal is to develop 
and extend Scotland's position as a world leader in Informatics and Computer Science research and education''~\cite{sicsa}. 
SICSA achieves this by working cooperatively rather than competitively, by providing support and sharing facilities, by working closely 
with industry and government and by appointing and retaining world-class staff and research students in Scottish Universities.
A list of members of SICSA is given below.

\begin{itemize}
 \item University of Aberdeen
 \item University of Abertay
 \item University of Dundee
 \item University of Edinburgh
 \item Edinburgh Napier University
 \item University of Glasgow
 \item Glasgow Caledonian University
 \item Heriot-Watt University
 \item Robert Gordon University
 \item University of St Andrews
 \item University of Stirling
 \item University of Strathclyde
 \item University of the West of Scotland
\end{itemize}

\section{What is Expert Search?}\label{sec:expertsearch}
With the enormous in the number of information and documents and the need to access information in large enterprise organisations,
``collaborative users regularly have the need to find not only documents, but also people with whom they share common interests, 
or who have specific knowledge in a required area''~\cite[P. 388]{expertsearch}. In an expert search task,
the users' need, expressed as queries, is to identify people who have relevant expertise to the need~\cite[P. 387]{expertsearch}.
An expert search system is an Information Retrieval~\cite{IR} system that makes use of textual evidence 
of expertise to rank candidates and can aid users with their ``expertise need''. Effectively, an expert search systems work by generating a ``profile'' 
of textual evidence for each candidate~\cite[P. 388]{expertsearch}. The profiles represent the system's knowledge of the expertise of each 
candidate, and they are ranked in response to a user query~\cite[P. 388]{expertsearch}. In real world scenario, the user formulates a query to 
represent their topic of interest to the system; the system then uses the available textual evidence of expertise to rank candidate persons with 
respect to their predicted expertise about the query.

\section{Definition of Mining Academic Expertise from Funded Research and Aims}\label{section:aims}
http://experts.sicsa.ac.uk/ ~\cite{sicsasearch} is a deployed academic search engine that assists in identifying the relevant experts within Scottish Universities, 
based on their recent publication output. However, integrating different kinds of academic expertise evidence with the existing one may improve
the effectiveness of the retrieval system. The aim of this project is to develop mining tools for the funded projects, 
and research ways to integrate them with the existing academic search engines to obtain the most effective search results. 
The sources of the new evidence, funded projects, are from Grant on the Web~\cite{gow} and Research Councils UK~\cite{gtr}.
To integrate academic funded projects and publications together, Learning to Rank Algorithms for Information Retrieval (IR) are applied in this project.

\section{Context}
This project was initially developed by an undergraduate student a few years ago. It used academic's publications as an expertise evidence to find experts.
I have access to funded projects data in the UK. This data is integrated with existing data to improve the performance of http://experts.sicsa.ac.uk/ ~\cite{sicsasearch}
The name of the project is \textit{AcademTech}. Some chapters might use this name.

\section{Overview}
In this dissertation, section~\ref{section:background} aims to explain the backgrounds of the project to readers. 
Section~\ref{sec:currentsystemandproposals} includes discussions of system and interface designs and architecture and proposals to the new system.
This section is necessary for readers to understand other sections. 
Section~\ref{section:implementation} discusses about implementations of the system and new system user interface design. 
Section~\ref{sec:evaluation} provides results and analysis of the techniques used. 
This section can be viewed as the most important part of the whole project since it 
analyses whether applying a learning to rank technique improves the performance of the system or not. 
The last section is the conclusions of the project.
