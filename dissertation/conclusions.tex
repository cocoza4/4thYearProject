
\chapter{Conclusions}
In this project, we applied 2 LETOR algorithms: AdaRank and Coordinate Ascent, to combine different kinds of expertise evidence in order 
to improve the performance of the system. There are various LETOR algorithms used in industry such as LambdaMART, Random Forests etc. 
However, AdaRank and Coordinate Ascent algorithms are chosen due to its simplicity and effective performance. Our experimental results (see last chapter) have shown that
\begin{itemize}
 \item AdaRank does not perform very well compared to Coordinate Ascent.
 \item Applying LETOR does not improve the performance of the system.
\end{itemize}

\noindent The reasons behind LETOR failure can be found in the previous chapter.

\section{Requirements Acheived}
The following is a summary of the requirements (see chapter~\ref{sec:requirements}) that have been achieved.
\begin{itemize}
 \item All of the system's Must Have requirements.
 \item All of the system's Should Have requirements.
 \item All of the non-functional Requirements.
\end{itemize}
The system's Would like to Have and Could Have requirements could not be acheived because of time constraint.
The Could Have requirements can be difficult for 4th year students~\cite{craig} as they are too complex and require full understanding of the 
behaviours of the technique used~\cite{craig}.
However, the aim of the project is to experiment 2 LETOR algorithms: AdaRank and Coordinate, and apply the one performing better.

\section{Challenges}\label{sec:challenges}
During the development of the project, we have encountered various challenges. This section will discuss challenges that I have encountered.

\subsection{LETOR}
Learning to rank is a relatively new
field in which machine learning algorithms are used to learn this ranking function~\cite{yahooLETOR}.
It is of particular importance for search engines to accurately tune their ranking functions as
it directly affects the search experience of users~\cite{yahooLETOR}.
Choosing the right LETOR technique is research based and it requires an 
understanding of the LETOR techniques behaviours which could possibly take weeks or months to come up with the best approach.

\subsection{Relevance Judgement}\label{sec:relevanceJudgement}
Relevance judgement (see chapter~\ref{sec:evaluation}) by human assessors is considered one of the most difficult parts in this project 
since it is based on person's experience~\cite{jjtextclassification}.
For this reason, constructing good quality training, testing and validating datasets is also a challenge as it directly affects the performance of 
the system~\cite{craig}.


\section{Problems Encountered}\label{sec:problems}
During the development of the project, we have encountered various problems. Some of them can be addressed and some can not. In this section, we 
list a number of problems encountered delelopment life cycle.

\subsection{Expert's Name Pattern Maching}
In chapter~\ref{sec:dataextraction}, we discussed approaches used to handle differences between expert's name formats in each source. 
Still, we have problems regarding pattern matching between expert's names from Grant on the Web~\cite{gow}. Consider Professor Joemon Jose,
he is currently a professor in computing science at Glasgow University. Grant on the Web records his name as Jose, Professor JM.
This makes it impossible to accurately pattern match the name with our known candidates if there are experts named Professor Joemon Jose and 
Professor Jasmine Jose because Grant on the Web will record their names in the same format: Jose, Professor JM. Although we could solve the problem by taking 
the university they are lecturering into account, the problem still exists if they are from the same university.

\subsection{Lack of LETOR Algorithms Explanation}
Clear explanations of the Coordinate Ascent and AdaRank LETOR algorithms are difficult to find. Trying to understand the nature of 
the algorithms is not an easy task for 4th year student as it is based on mathematics and statistics.

\subsection{Poor Quality of Relevance Judgements}
In chapter~\ref{sec:evaluation}, we talked about the problems we encountered when constructing relevance judgements. That is,
we managed to construct training queries whose results (relevant experts) are only from University of Glasgow (see last chapter). 
The reason behind this is that it is difficult to judge experts not in University of Glasgow because we do not know them.
As a consequence of poor relevance judgement, the dataset used for evaluation becomes poor~\cite{craig}.


\section{How would I have done differently?}
In sections~\ref{sec:challenges} and~\ref{sec:problems}, we discussed challenges and problems encountered during the development of the system.
However, these issues could be minimised or eliminated if 
\begin{itemize}
 \item The behaviours of LETOR algorithms is well understood at the early stage of the development - this means that we could experiment only 
 suitable algorithms whose behaviour is applicable to the task.
 \item Various funded projects datasources should be used to increase the number of expertise evidence - this means that we would have more training/validation
 data.
 \item Relevance Judgements are prepared by experienced persons - this means that the performance of LETOR techniques could be enhanced as relevance judgements
 are used to produce training and validation datasets.
 \item Tools used in this project are familiarised at the early stage of the development - this means that we could focus more on other aspects of the 
 project.
\end{itemize}


\section{Future Work}
In the previous chapter, the experimental results have shown that applying LETOR techniques using AdaRank and Coordinate Ascent algorithms 
does not improve the performance
of the system. This might not be true however, if the system applies other LETOR algorithms provided by RankLib since each algorithm has its own behaviours.
In addition, good training/validation datasets and feature selections play the main roles to the performance of the system~\cite{craig}. I strongly believe that the intuitions (features)
proposed in section~\ref{section:goodexpert} are very useful but the main reason why LETOR fails is due to small training dataset. 
We had 34 training queries which is very small. 

Moreover, for IR systems to get better performance, not only LETOR technique is contributed to the success of the retrieval system, but also 
other aspects in IR such as tokenisation technique, stemming technique and retrieval models.
LETOR is just an optional technique in machine learning used to optimise the ranking based 
on training/validation datasets. However, all of these aspects should be experimented.
